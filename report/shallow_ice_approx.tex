In this section, we will take into account the specific geometry of a glacier.  We will do so by rescaling the equations in an appropriate way and identifying (and later ignoring) small terms, using as a main basis the fact that the length $L$ of a glacier is typically much larger than its height $H$. We therefore choose the scales
%
\begin{equation*}
    x^* = Lx \quad \text{and} \quad z^*=Hz
\end{equation*}
%
and try to make use of the fact that
%
\begin{equation*}
    \epsilon \coloneqq \frac{H}{L} \ll 1.
\end{equation*}
%
We scale the height with the same scale $H$;
%
\begin{equation*}
    h = Hh^*.
\end{equation*}
