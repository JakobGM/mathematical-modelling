In the previous asymptotic expansions of equations~\eqref{eq:interior_equations_rescaled} we did not treat $\alpha$ as a small parameter in the same way as $\epsilon$. If we assume $\gamma = \epsilon \cot(\alpha) \sim 1$, we still obtain a relatively simple PDE for $h$.

With the new assumption about $\alpha$, first order asymptotic asymptotic expansion of equation~\eqref{eq:interior_equations_rescaled1} gives $\partial_z \tau_{xz} = - \partial_x h$. Thus $\tau_{xz} = h_x (h-z)$, $u = \frac{\kappa}{m+1} h_x^m \left[ h^{m+1} - (h-z)^{m+1} \right]$, and finally we arrive at \textcolor{red}{(assuming what we assume in Problem 8) check}

\begin{equation}
\pd{h}{t} + \lambda \od{}{x} \left(h_x^m \, \red{(}h\red{-d(x))}^{m+2}\right) = q.
\end{equation}

\red{Assuming that $h_x^m$ is treated as a constant. Maybe we can just say that from here we restrict ourselves to modelling where $d(x) \equiv 0$. Consequences? -Johan}