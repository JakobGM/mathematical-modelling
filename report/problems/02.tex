Conservation of momentum and Newton's second law of motion implies that the velocity of the glacier must satisfy

\begin{equation} \label{eq:momentum-conservation}
  \begin{cases}
    &\rho \partial_{t^*} u^* + \rho \nabla \red{(}u^* \cdot \vec{w}^*\red{)} = -\partial_{x^*} p^* + \nabla \cdot \tau_x^* + f_x, \\
    &\rho \partial_{t^*} v^* + \rho \nabla \red{(} v^* \cdot \vec{w}^*\red{)} = -\partial_{z^*} p^* + \nabla \cdot \tau_z^* + f_z,
  \end{cases}
\end{equation}

\red{We should convince ourselves that those red parens should not be there!}

Here, $f = (f_x, f_y)$ denotes the body forces on each point, $p^*$ the pressure, and

\begin{equation}
  \tau^* = \begin{bmatrix} \tau_x^* \\ \tau_z^* \end{bmatrix} = \begin{bmatrix} \tau_{xx}^* & \tau_{xz}^* \\ \tau_{zx}^* & \tau_{zz}^* \end{bmatrix}
\end{equation}

the stress tensor in two dimensions. The pressure $p^*$ can be written as a sum

\begin{equation} \label{eq:pressure}
  p^*(x^*, z^*, t^*) = \rho g \cos(\alpha) \cdot (h^*(x^*, t^*) - z^*) + \tilde{p}^*(x^*, z^*, t^*)
\end{equation}

with $g$ being the gravitational acceleration at the surface of the earth ($\approx \SI{9.8}{\meter\per\square\second}$). The first term in \eqref{eq:pressure} is the hydrostatic pressure within the glacier, and the second term is the counterpressure induced by the constant density $\rho$.

We will now model the body forces within the glacier \red{(what does "model" entail exactly here?)}. First observe that the two first terms of the left hand side of equation \eqref{eq:momentum-conservation} are the time and space derivatives of momentum.
We can make a simplification by assuming that gravity and friction changes are much greater than momentum and pressure changes, resulting in a steady state. This two terms can therefore be approximated to $0$, simplifying equation \eqref{eq:momentum-conservation} to \red{Is this correct? Pressure should probably not be neglected...}

\begin{equation} \label{eq:steady-state-unknown-force}
  \begin{cases}
    &-\partial_{x^*} p^* + \nabla \cdot \tau_x^* + f_x = 0, \\
    &-\partial_{z^*} p^* + \nabla \cdot \tau_z^* + f_z = 0,
  \end{cases}
\end{equation}

The pointwise gravitational force $\vec{f_g}$ is given by

\begin{equation}
  \vec{f_g} = \begin{bmatrix} f_{gx} \\ f_{gz} \end{bmatrix} = \begin{bmatrix} \rho g \sin(\alpha) \\ \rho g \cos(\alpha) \end{bmatrix}.
\end{equation}

Under the assumption that the dominating body force is $f_g$ \red{(again, this needs to be argued for better)}, equation \eqref{eq:steady-state-unknown-force} becomes:

\begin{equation} \label{eq:steady-state-implicit}
  \begin{cases}
    &-\partial_{x^*} p^* + \nabla \cdot \tau_x^* + \rho g \sin(\alpha) = 0, \\
    &-\partial_{z^*} p^* + \nabla \cdot \tau_z^* + \rho g \cos(\alpha) = 0,
  \end{cases}
\end{equation}

The space derivatives of the pressure sum given in equation \eqref{eq:pressure} can be calculated as

\begin{equation} \label{eq:pressure-space-derivative}
  \begin{align}
  \partial_{x^*} p^* &= \rho g \cos(\alpha) \cdot \partial_{x^*} h^* + \partial_{x^*} \tilde{p}^*, \\
  \partial_{z^*} p^* &= - \rho g \cos(\alpha) + \partial_{z^*} \tilde{p}^*.
  \end{align}
\end{equation}

Inserting these results into equation \eqref{eq:steady-state-implicit} yields

\begin{equation} \label{eq:steady-state-explicit}
  \begin{align}
    \nabla \cdot \tau_x^* + \rho g \sin(\alpha) - \rho g \cos(\alpha) \cdot \partial_{x^*} h^* - \partial_{x^*} \tilde{p}^* &= 0, \\
    \nabla \cdot \tau_z^* - \partial_{z^*} \tilde{p}^* &= 0,
  \end{align}
\end{equation}
